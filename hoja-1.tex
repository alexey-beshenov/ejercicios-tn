\documentclass{article}

\usepackage{fullpage}

\usepackage[utf8]{inputenc}
\usepackage[spanish]{babel}

\usepackage[T1]{fontenc}
\usepackage{fourier}
\usepackage{paratype}

\usepackage{framed}
\usepackage{hyperref}
\usepackage{amsmath,amssymb}
\usepackage{tikz}

\newcommand{\ZZ}{\mathbb{Z}}
\DeclareMathOperator{\mcd}{mcd}
\DeclareMathOperator{\mcm}{mcm}
\DeclareMathOperator{\ord}{ord}

\usepackage{amsthm}
\theoremstyle{definition}
\newtheorem{problema}{Problema}[section]
\newtheorem*{comentario}{Comentario}
\newtheorem*{definicion}{Definición}
\newtheorem*{ejemplo}{Ejemplo}

\newenvironment{solucion}{\begin{proof}[Solución]\small}{\end{proof}}

\renewcommand*{\thefootnote}{\fnsymbol{footnote}}


\title{Hoja 1: Residuos invertibles mód $n$}
\author{Alexey Beshenov (cadadr@gmail.com)}
\date{23 de septiembre de 2021}

\begin{document}

\maketitle
\thispagestyle{empty}

\setcounter{section}{1}

\begin{definicion}
  Se dice que un residuo $x \in \ZZ/n\ZZ$ es \textbf{invertible} si existe otro
  residuo $y \in \ZZ/n\ZZ$ tal que $xy = 1$. En este caso también se escribe
  $y = x^{-1}$.

  De manera equivalente, $a \in \ZZ$ es \textbf{invertible mód $n$} si existe
  $b \in \ZZ$ tal que $ab \equiv 1 \pmod{n}$.
\end{definicion}

El conjunto de los residuos invertibles módulo $n$ se denotará por
$(\ZZ/n\ZZ)^\times$.

\begin{ejemplo}
  He aquí los residuos módulo $n = 15$ y sus inversos; <<--->> significa que el
  residuo no es invertible.

  \begin{center}
    \begin{tabular}{ccccccccccccccc}
      \hline
      $0$ & $1$ & $2$ & $3$ & $4$ & $5$ & $6$ & $7$ & $8$ & $9$ & $10$ & $11$ & $12$ & $13$ & $14$ \\
      \hline
      --- & $1$ & $8$ & --- & $4$ & --- & --- & $13$ & $2$ & --- & --- & $11$ & --- & $7$ & $14$ \\
      \hline
    \end{tabular}
  \end{center}
\end{ejemplo}

\begin{problema}
  Demuestre que si $x, y \in (\ZZ/n\ZZ)^\times$ son residuos invertibles mód
  $n$, entonces $x^{-1}$ y $xy$ son también invertibles.
\end{problema}

\begin{problema}[Cancelación]
  Demuestre que si $x,y,z$ son residuos módulo $n$, y $z$ es invertible,
  entonces $xz = yz$ implica que $x = y$. ¿Qué pasa si $z$ no es invertible?
\end{problema}

\begin{problema}
  \label{probl:invertible-coprimo}
  En este problema vamos a probar que $[a]_n$ es invertible si y solo si
  $\mcd (a,n) = 1$:
  \[ (\ZZ/n\ZZ)^\times = \{ [a]_n \mid 0 \le a < n, ~ \mcd (a,n) = 1 \}. \]

  \begin{enumerate}
  \item[a)] Si $\mcd (a,n) = 1$, use la identidad de Bézout para encontrar
    $[a]_n^{-1}$.

  \item[b)] Demuestre que si $x,y \in \ZZ/n\ZZ$ son dos residuos no nulos tales
    que $xy = 0$, entonces $x$ e $y$ no pueden ser invertibles.

    (En otras palabras, si $n \mid ab$ para $n \nmid a$, $n \nmid b$, entonces
    $a$ y $b$ no son invertibles módulo $n$.)

  \item[c)] Si $\mcd (a,n) > 1$, use el punto anterior para probar que $a$ no es
    invertible mód $n$.
  \end{enumerate}
\end{problema}

\begin{problema}
  Para $n = 5, 6, 7, 8$ encuentre cuáles residuos mód $n$ son invertibles y
  escriba sus inversos correspondientes.
\end{problema}

\begin{problema}
  Calcule $[6]_{385}^{-1}$.
\end{problema}

\begin{problema}[Teorema de Wilson]
  \label{probl:Wilson-1}
  Sea $p$ un primo.

  \begin{enumerate}
  \item[a)] Demuestre que para todo $x \in (\ZZ/p\ZZ)^\times$ se tiene
    $x^{-1} = x$ si y solamente si $x = \pm 1$.

  \item[b)] Deduzca de a) que $(p - 1)! \equiv -1 \pmod{p}$.

  \item[c)] Demuestre que $(n-1)! \equiv 0 \pmod{n}$ si $n \ge 6$ es compuesto.
  \end{enumerate}
\end{problema}

\begin{problema}
  Demuestre que para todo primo impar $p$, el numerador de
  $$1 + \frac{1}{2} + \frac{1}{3} + \cdots + \frac{1}{p-1}$$
  es divisible por $p$.

  Sugerencia: $1, 2, \ldots, p-1$ son invertibles mód $p$.
\end{problema}

\end{document}
