\documentclass{article}

\usepackage{fullpage}

\usepackage[utf8]{inputenc}
\usepackage[spanish]{babel}

\usepackage[T1]{fontenc}
\usepackage{fourier}
\usepackage{paratype}

\usepackage{hyperref}
\usepackage{amsmath,amssymb}
\usepackage{tikz}

\newcommand{\ZZ}{\mathbb{Z}}
\DeclareMathOperator{\mcd}{mcd}

\usepackage{amsthm}
\theoremstyle{definition}
\newtheorem{problema}{Problema}
\newtheorem*{comentario}{Comentario}

\title{Algunos ejercicios de la teoría de números elemental}
\author{Alexey Beshenov (cadadr@gmail.com)}
\date{21 de septiembre de 2021}

\newenvironment{solucion}{\begin{proof}[Solución]\small}{\end{proof}}

\begin{document}

\maketitle

Los ejercicios de esta hoja usan los siguientes conceptos de la teoría de
números elemental: divisibilidad, mcd, identidad de Bézout, factorización en
números primos, congruencias mód $n$, función $\phi$ de Euler.

\ifdefined\solutions
\else
\vspace{1em}

Para las soluciones, véase

\url{https://cadadr.org/teoria-de-numeros-basica/guia-1-soluciones.pdf}
\fi

\section*{Ejercicios para hacer en vivo}

\setcounter{problema}{-1}
\begin{problema}
  ~
  \begin{enumerate}
  \item[a)] Encuentre el mínimo número positivo $n$ de la forma $30\,a + 105\,b$
    para $a,b \in \ZZ$.

    ¿Cuáles son los $a$ y $b$ correspondientes? ¿Son únicos?

  \item[b)] Calcule $5^{2021} \pmod{13}$.

  \item[c)] Para $N = 105$, ¿cuántos números $0 \le x < N$ son invertibles
    módulo $N$?

  \item[d)] Para $n = 2,3,4,5,6$, ¿cuántos números $0 \le x < 13$ satisfacen
    $x \equiv y^n \pmod{13}$ para algún $y \in \ZZ$?
  \end{enumerate}

  \ifdefined\solutions\begin{solucion}
    En a), tenemos $\mcd (30,105) = 15$, así que $30\,a + 105\,b$ es divisible
    por $15$. Esto demuestra que $30\,a + 105\,b \ge 15$, cuando $30\,a + 105\,b$
    es positivo. El algoritmo de Euclides nos da la identidad de Bézout con
    $(a,b) = (-3,1)$:
    $$30\cdot (-3) + 105\cdot 1 = 15.$$
    Los coeficientes $a,b$ no son para nada únicos: por ejemplo, en lugar de
    $(-3,1)$ funcionaría $(4,-1)$. Véase el problema $12$.

    \vspace{1em}

    En b), notamos que $5^2 = 25 \equiv 12 \equiv -1 \pmod{13}$. Luego,
    \[
      5^3 \equiv -5, ~
      5^4 \equiv 1, ~
      \ldots, ~
      5^{2021} = 5\cdot (5^4)^{505} \equiv 5.
    \]

    En c), el número de los restos invertibles mód $n$ es la función de Euler
    $\phi(n)$. Luego
    $\phi (105) = \phi(3)\,\phi(5)\,\phi(7) = 2\cdot 4\cdot 6 = 48$.

    \vspace{1em}

    En d), podemos usar el siguiente resultado poderoso: para todo primo $p$
    existe un número $x \in \ZZ$ (llamado una \textbf{raíz primitiva mód $p$})
    tal que las potencias
    $$1, x, x^2, \ldots, x^{p-2}$$
    representan los diferentes restos no nulos módulo $p$.

    En general no hay una manera sencilla de encontrar el $x$ en cuestión, pero
    para $p = 13$ funciona $x = 2$:
    \[
      2^0 \equiv 1, ~
      2^1 \equiv 2, ~
      2^2 \equiv 4, ~
      2^3 \equiv 8, ~
      2^4 \equiv 3, ~
      2^5 \equiv 6, ~
      2^6 \equiv 12, ~
      2^7 \equiv 11, ~
      2^8 \equiv 9, ~
      2^9 \equiv 5, ~
      2^{10} \equiv 10, ~
      2^{11} \equiv 7.
    \]

    Luego los cuadrados no nulos mód $13$ son las potencias pares de $2$:
    \[ 1 \equiv 2^0, ~
      3 \equiv 2^4, ~
      4 = 2^2, ~
      9 \equiv 2^8, ~
      10 \equiv 2^{10}, ~
      12 \equiv 2^6. \]
    De la misma manera, las cuartas potencias no nulas son
    \[ 1 \equiv 2^0, ~
      3 \equiv 2^4, ~
      9 \equiv 2^8. \]
    Los cubos serán las potencias de $2$ divisibles por $3$:
    \[ 1 \equiv 2^0, ~
      5 \equiv 2^9, ~
      8 = 2^3, ~
      12 \equiv 2^6. \]    
    Aquí las sextas potencias son $1$ y $12 \equiv -1$.

    Lo más interesante sucede con las quintas potencias. Tenemos $2^5 = 6$, y de
    hecho $6$ es otra raíz primitiva: las potencias de $6$ recorren todos los
    restos no nulos mód $13$. Esto significa que cualquier resto es una quinta potencia:
    \[
      2 \equiv 6^5, ~
      3 \equiv 9^5, ~
      4 \equiv 10^5, ~
      5 \equiv 5^5, ~
      6 \equiv 2^5, ~
      7 \equiv 11^5, ~
      8 \equiv 8^5, ~
      9 \equiv 3^5, ~
      10 \equiv 4^5, ~
      11 \equiv 7^5, ~
      12 \equiv 12^5.
    \]
    Esto se debe al hecho de que $\mcd (5,12) = 1$. (¿Por qué?)
  \end{solucion}\fi
\end{problema}

\begin{problema}
  Si $\mcd (a,b) = \mcd (c,d) = 1$ y $\frac{a}{b} + \frac{c}{d}$ es un número
  entero, demuestre que necesariamente $b = \pm d$.

  \ifdefined\solutions\begin{solucion}
    Tenemos
    $$\frac{a}{b} + \frac{c}{d} = \frac{ad + bc}{bd},$$
    donde
    $$ad + bc = bde.$$
    Ahora $ad = b\,(de - c)$ y $\mcd (a,b) = 1$, así que $b \mid d$. De manera
    similar se ve que $d \mid b$.
  \end{solucion}\fi
\end{problema}

\begin{problema}
  El teorema de Wilson afirma que $(p - 1)! \equiv -1 \pmod{p}$ para $p$
  primo. Demuestre que $(n-1)! \equiv 0 \pmod{n}$ si $n \ge 6$ es compuesto.

  \ifdefined\solutions\begin{solucion}    
    Supongamos que $n = ab$, donde $1 < a,b < n$. Entre los números
    $2,3,4,\ldots,n-1$ necesariamente aparecen $a$ y $b$. Si $a \ne b$, esto nos
    permite concluir que $n \mid (n-1)!$. En el caso especial de $a = b$,
    tenemos $n = a^2$. Notamos que $2a < a^2$ para $a > 2$, así que entre los
    números $2,3,4,\ldots,a^2-1$ aparecen $a$ y $2a$. Luego, $a^2 \mid (a^2-1)!$
    para $a > 2$.
  \end{solucion}\fi
\end{problema}

\begin{problema}
  ~

  \begin{enumerate}
  \item[a)] Demuestre que $\mcd(a, a + b) \mid b$.

  \item[b)] Si $\mcd (a,b) = 1$, demuestre que $\mcd (a+b, a-b) = 1$ o $2$.

    En particular, calcule $\mcd (a+1, a-1)$.
  \end{enumerate}

  \ifdefined\solutions\begin{solucion}
    En parte a), si $d = \gcd (a,a+b)$, entonces $d \mid a$ y $d \mid a+b$, y
    luego $d \mid b$.

    \vspace{1em}

    En la parte b), tenemos $d \mid (a+b)$ y $d \mid (a-b)$, así que $d \mid 2a$
    y $d \mid 2b$. Puesto que $a$ y $b$ son coprimos, podemos concluir que
    $d = 1$ o $2$.

    En particular, no es difícil calcular que
    \[ \mcd (a-1,a+1) =
      \begin{cases}
        1, & \text{si }a\text{ es par}, \\
        2, & \text{si }a\text{ es impar}.
      \end{cases} \]
    \end{solucion}\fi
\end{problema}

\begin{problema}[IWYMIC 2019 Individual contest]
  Encuentre el mínimo $n$ tal que $x = 55 n^3$ que tiene $55$ divisores
  $1 \le d \le x$, $d \mid x$.

  \ifdefined\solutions\begin{solucion}
    Vamos a denotar por $d (x)$ la función del número de divisores. Tenemos
    $d (p^e) = e+1$. Esta función es multiplicativa, así que
    $$d (p_1^{e_1}\cdots p_s^{e_s}) = (e_1+1)\cdots (e_s+1).$$
    En nuestro caso, buscamos $n$ tal que $d (55\,n^3) = 55$. Puesto que
    $55 = 5\cdot 11$, no tenemos muchas opciones:
    $x = p^4\,q^{10}$ o $p^{10}\,q^4$, donde $p = 5$, $q = 11$.
    El número más pequeño es visiblemente $5^{10}\cdot 11^4$. Sustituyendo
    $n = 5^a\cdot 11^b$, calculamos de
    $$5^{10}\cdot 11^4 = 55\,n^3 = 5^{3a+1}\cdot 11^{3b+1}$$
    que $n = 5^3\cdot 11 = 1375$.
  \end{solucion}\fi
\end{problema}

\begin{problema}[IWYMIC 2019 Team contest]
  $p_1, p_2, p_3$ son primos tales que
  $$p_1 p_2 p_3 = p_1 + p_2 + p_3 + 1007.$$
  Encuentre $p_1$, $p_2$, $p_3$.

  \ifdefined\solutions\begin{solucion}
    Reduciendo mód 2, primero notamos que necesariamente dos de estos primos son
    pares: $p_1 = p_2 = 2$. Pongamos $p_3 = p$. Nos queda nada más
    $4p = p + 1011$, de donde $p = 337$. Este número es primo.
  \end{solucion}\fi
\end{problema}

\begin{problema}
  Se fija un punto en el plano y se consideran unas copias de $n$-ágono
  regular, tratando de colocarlas alrededor del punto.
  Demuestre que esto es posible solamente con $6$ triángulos, o $3$ cuadrados, o
  $6$ hexágonos, como en el dibujo de abajo:

  \begin{center}
    \begin{tikzpicture}
      \node at (0,0) [circle,fill,inner sep=1pt]{};

      \draw (0:0)--(0:1)--(60:1);
      \draw (0:0)--(60:1)--(120:1);
      \draw (0:0)--(120:1)--(180:1);
      \draw (0:0)--(180:1)--(240:1);
      \draw (0:0)--(240:1)--(300:1);
      \draw (0:0)--(300:1)--(360:1);

      \node at (3,0) [circle,fill,inner sep=1pt]{};

      \draw[xshift=3cm] (0,0)--(1,0)--(1,1)--(0,1);
      \draw[xshift=3cm] (0,0)--(0,1)--(-1,1)--(-1,0);
      \draw[xshift=3cm] (0,0)--(-1,0)--(-1,-1)--(0,-1);
      \draw[xshift=3cm] (0,0)--(0,-1)--(1,-1)--(1,0);

      \node at (7,0) [circle,fill,inner sep=1pt]{};

      \draw[xshift=8cm] (120:1) -- (60:1) -- (0:1) -- (300:1) -- (240:1);

      \draw[xshift=6.5cm,yshift=0.86602cm] (0:1) -- (60:1) -- (120:1) -- (180:1) -- (240:1);
      \draw[xshift=6.5cm,yshift=-0.86602cm] (120:1) -- (180:1) -- (240:1) -- (300:1) -- (360:1);
      \draw[xshift=7cm] (0,0) -- (60:1);
      \draw[xshift=7cm] (0,0) -- (180:1);
      \draw[xshift=7cm] (0,0) -- (300:1);
    \end{tikzpicture}
  \end{center}

  Sugerencia: si $N$ es el número de $n$-ágonos, entonces
  $N\cdot (n-2)\,\frac{\pi}{n} = 2\pi$. (¿Por qué?)

  \ifdefined\solutions\begin{solucion}

    Un ángulo del $n$-ágono regular mide $(n-2)\,\frac{\pi}{n}$. Por otra parte,
    la suma de $N$ estos ángulos alrededor del punto debe ser $2\pi$. Tenemos
    entonces $N\cdot (n-2)\,\frac{\pi}{n} = 2\pi$. Esto nos deja la ecuación
    $N = \frac{2n}{n-2}$. Hay que ver que $(n,N) = (3,6), (4,4), (6,3)$ son las
    únicas soluciones. La función $f (n) = \frac{2n}{n-2}$ decrece para
    $n \ge 3$, y además $f (n) \xrightarrow{n\to \infty} 2$. Calculamos que
    $$f (3) = 6, \quad f (4) = 4, \quad f (5) = \frac{10}{3}, \quad f (6) = 3, \quad f (7) = \frac{14}{5} < 3,$$
    así que $2 < f (n) < 3$ para $n \ge 7$. Entonces, los únicos valores enteros
    son $f (3) = 6$, $f (4) = 4$, $f (6) = 3$.
  \end{solucion}\fi
 \end{problema}

 \begin{problema}
   Demuestre que si $a^n - 1$ es primo, entonces $a = 2$ y $n$ es primo.

   \ifdefined\solutions\begin{solucion}
     Si $p \mid n$ para un primo impar $p$, entonces escribiendo $m = n/p$,
     se obtiene
     $$a^n - 1 = (a^m)^p + 1^p = (a^m + 1)\,(a^{m\,(p-1)} + a^{m\,(p-2)} + \cdots + 1),$$
     y necesariamente $a = 2$ y $n = p$.

     Un lema muy útil: para $a > 2$ tenemos $a^m - 1 \mid a^n - 1$ si y
     solamente si $m \mid n$ (¡demuéstrelo!)
   \end{solucion}\fi
\end{problema}

\begin{comentario}
  Los primos de la forma $2^p - 1$ se conocen como los
  \textbf{primos de Mersenne}; su infinitud es una conjetura abierta. Muchos de
  los números $2^p - 1$ son compuestos, el primero siendo
  $2^{11} - 1 = 23\cdot 89$.

  \begin{center}
    \begin{tabular}{ll}
      $2^2 - 1\colon$ & primo $= 3$ \\
      $2^3 - 1\colon$ & primo $= 7$ \\
      $2^5 - 1\colon$ & primo $= 31$ \\
      $2^7 - 1\colon$ & primo $= 127$ \\
      $2^{11} - 1\colon$ & compuesto $= 23\cdot 89$ \\
      $2^{13} - 1\colon$ & primo $= 8191$ \\
      $2^{17} - 1\colon$ & primo $= 131071$ \\
      $2^{19} - 1\colon$ & primo $= 524287$ \\
      $2^{23} - 1\colon$ & compuesto $= 47\cdot 178481$ \\
      $2^{29} - 1\colon$ & compuesto $= 233\cdot 1103\cdot 2089$ \\
      $2^{31} - 1\colon$ & primo $= 2147483647$ \\
      $2^{37} - 1\colon$ & compuesto $= 223\cdot 616318177$ \\
      $2^{41} - 1\colon$ & compuesto $= 13367\cdot 164511353$ \\
                      & $\cdots$ \\
      $2^{61} - 1\colon$ & primo $= 2305843009213693951$ \\
      $2^{67} - 1\colon$ & compuesto $= 193707721\cdot 761838257287$ \\
                      & $\cdots$
    \end{tabular}
  \end{center}
\end{comentario}

\begin{problema}
  Demuestre que si $a^n + 1$ es primo, entonces $a$ es par y $n = 2^k$ es
  una potencia de $2$.

%   \ifdefined\solutions\begin{solucion}
%     If $a$ is odd, then $a^n + 1$ is even, so it's clearly not a prime. This shows that $a$ must be even. Now consider an odd prime $p \mid n$ and let $m = n/p$. Then
% $$a^n + 1 = (a^m)^p + 1^p = (a^m + 1)\,(a^{m\,(p-1)} - a^{m\,(p-2)} + a^{m\,(p-3)} - \cdots - a^m + 1).$$
% We note that the alternating sum in the second multiple is greater than $1$, and the above factorization means that $a^n + 1$ is not prime. This shows that no odd prime $p$ can divide $n$, and therefore $n = 2^t$ for some $t$.
%   \end{solucion}\fi
\end{problema}

\begin{comentario}
  Los primos de la forma $2^{2^k} + 1$ se conocen como los
  \textbf{primos de Fermat}. Su finitud es una conjetura abierta.
  El primer número compuesto de esta forma es $2^{2^5}+1 = 641\cdot 6700417$
  (dos factores primos). Note que estos números crecen muy rápido con $k$.

    \begin{center}
      \begin{tabular}{ll}
        $2^{2^0} + 1\colon$ & primo $= 3$ \\
        $2^{2^1} + 1\colon$ & primo $= 5$ \\
        $2^{2^2} + 1\colon$ & primo $= 17$ \\
        $2^{2^3} + 1\colon$ & primo $= 257$ \\
        $2^{2^4} + 1\colon$ & primo $= 65537$ \\
        $2^{2^5} + 1\colon$ & compuesto $= 641\cdot 6700417$ \\
        $2^{2^6} + 1\colon$ & compuesto $= 274177\cdot 67280421310721$ \\
        $2^{2^7} + 1\colon$ & compuesto $= 59649589127497217\cdot 5704689200685129054721$
      \end{tabular}
    \end{center}

    Fermat conjeturó (demasiado optimísticamente) que los números de la forma
    $2^{2^k} + 1$ son primos, pero Euler descubrió que
    $641 \mid (2^{2^5} + 1)$. De hecho, parece que para todo $k \ge 5$ los
    números $2^{2^k} + 1$ son compuestos.
\end{comentario}

\ifdefined\solutions
\else
\pagebreak
\fi

\begin{problema}
  ~

  \begin{enumerate}
  \item[a)] Para $a \ne 0$ y $m \ne n$ calcule $\mcd (a^{2^m} + 1, a^{2^n} + 1)$.

  \item[b)] Concluya que los números $2^2 + 1$, $2^{2^2} + 1$, $2^{2^3} + 1$,
    $\ldots$ son coprimos por pares.
  \end{enumerate}
\end{problema}

\pagebreak

\section*{Ejercicios para hacer en casa}

\begin{problema}
  ~

  \begin{enumerate}
  \item[a)] Demuestre que si $n$ es impar, entonces $8 \mid (n^2 - 1)$. Además,
    si $3\nmid n$, entonces $6 \mid (n^2 - 1)$.

  \item[b)] Demuestre que para todo $n$, se tiene $30 \mid (n^5 - n)$ y
    $42 \mid (n^7 - n)$.
  \end{enumerate}
\end{problema}

\begin{problema}
  Sean $a,b,c$ números enteros.

  \begin{enumerate}
  \item[a)] Demuestre que la ecuación
    $$ax + by = c$$
    tiene una solución entera $(x,y)$ si y solamente si $\mcd (a,b) \mid c$.

  \item[b)] (*) Si $(x_0,y_0)$ es una solución, demuestre que todas las
    soluciones tienen forma
    $$x = x_0 + t\,\frac{b}{d}, \quad y = y_0 - t\,\frac{a}{d},$$
    para $d = \mcd (a,b)$ y $t \in \ZZ$.
  \end{enumerate}
\end{problema}

\begin{problema}
  Para un entero $n$ y un parámetro natural $k = 0,1,2,\ldots$, definamos
  $$\sigma_k (n) = \sum_{\substack{1 \le d \le n \\ d \mid n}} d^k.$$

  \begin{enumerate}
  \item[a)] Demuestre que $\sigma_k (mn) = \sigma_k (m) \, \sigma_k (n)$ para
    $\mcd (m,n) = 1$.

  \item[b)] Para $n = p_1^{e_1} \cdots p_s^{e_s}$, deduzca una fórmula para
    $\sigma_k (n)$.
  \end{enumerate}
\end{problema}

\begin{problema}
  Como un caso particular del ejercicio anterior, consideremos el número de
  divisores
  $$d (n) = \sigma_0 (n) = \# \{ 1 \le d \le n \mid d \mid n \}.$$

  Encuentre los números tales que $d (n) = n/3$ y $n = d(n)^2$.
\end{problema}

\begin{problema}[IWYMIC 2019 Individual contest]
  Encuentre todas las soluciones enteras $(m,n)$ de la ecuación
  $$\frac{m^2 + mn + n^2}{m + 2n} = \frac{13}{3}.$$
\end{problema}

\begin{problema}[IWYMIC 2019 Team contest]
  Encontrar todos los posibles dígitos $(a,b)$ (es decir, $0 \le a, b \le 9$)
  tales que el número $2a1b9$ cumple la congruencia
  $$2a1b9^{2019} \equiv 1 \pmod{13}.$$
\end{problema}

\end{document}
