\documentclass{article}

\usepackage{fullpage}

\usepackage[utf8]{inputenc}
\usepackage[spanish]{babel}

\usepackage[T1]{fontenc}
\usepackage{fourier}
\usepackage{paratype}

\usepackage{framed}
\usepackage{hyperref}
\usepackage{amsmath,amssymb}
\usepackage{tikz}

\newcommand{\ZZ}{\mathbb{Z}}
\DeclareMathOperator{\mcd}{mcd}
\DeclareMathOperator{\mcm}{mcm}
\DeclareMathOperator{\ord}{ord}

\usepackage{amsthm}
\theoremstyle{definition}
\newtheorem{problema}{Problema}[section]
\newtheorem*{comentario}{Comentario}
\newtheorem*{definicion}{Definición}
\newtheorem*{ejemplo}{Ejemplo}

\newenvironment{solucion}{\begin{proof}[Solución]\small}{\end{proof}}

\renewcommand*{\thefootnote}{\fnsymbol{footnote}}


\title{Hoja 0: Residuos mód $n$ (breve repaso)}
\author{Alexey Beshenov (cadadr@gmail.com)}
\date{23 de septiembre de 2021}

\begin{document}

\maketitle

\setcounter{section}{2}

\begin{definicion}
  Fijemos algún $n = 1,2,3,\ldots$ y consideremos la siguiente relación sobre
  los números enteros: se dice que $a$ y $b$ son \textbf{congruentes módulo $n$}
  si $n$ divide a $a-b$:
  $$a \equiv b \pmod{n} \iff n \mid (a-b).$$
  En otras palabras, $a$ y $b$ tienen el mismo residuo de la división por $n$.
\end{definicion}

\begin{problema}
  Demuestre que la congruencia mód $n$ es una relación de equivalencia:
  para cualesquiera $a,b,c\in \ZZ$ se tiene
  \[
    a\equiv a, \quad
    a\equiv b \Rightarrow b\equiv a, \quad
    a\equiv b \text{ y } b\equiv c \Rightarrow a\equiv c.
  \]
\end{problema}

\begin{definicion}
  Las clases de equivalencia se llaman los \textbf{residuos módulo
    $n$}\footnote{O también \textbf{residuos módulo $n$}.}. La clase de
  equivalencia de $a$ será denotada por $[a]_n$, o simplemente por $[a]$:
  $$[a]_n = [b]_n \iff a\equiv b \pmod{n}.$$
  
  El conjunto de los residuos mód $n$ se denota por $\ZZ/n\ZZ$. Note que este
  tiene precisamente $n$ elementos, representados por los posibles residuos de
  división por $n$:
  $$\ZZ/n\ZZ = \{ [0], \, [1], \, \ldots, \, [n-1] \}.$$
\end{definicion}

\begin{problema}
  Demuestre que si $a \equiv a'$, $b \equiv b'$, entonces
  \[ a+b \equiv a'+b', \quad a\cdot b \equiv a'\cdot b'. \]

  Esto quiere decir que la adición y multiplicación tiene sentido para los
  residuos mód $n$: podemos definir
  \[ [a] + [b] = [a+b], \quad [a]\cdot [b] = [a\cdot b]. \]
\end{problema}

En lugar de $[0]_n$ y $[1]_n$ será conveniente escribir simplemente $0$ y $1$.

\begin{ejemplo}
  He aquí las tablas de adición y multiplicación módulo $5$:
  \[
    \begin{array}{c|ccccc}
      + & 0 & 1 & 2 & 3 & 4 \\
      \hline
      0 & 0 & 1 & 2 & 3 & 4 \\
      1 & 1 & 2 & 3 & 4 & 0 \\
      2 & 2 & 3 & 4 & 0 & 1 \\
      3 & 3 & 4 & 0 & 1 & 2 \\
      4 & 4 & 0 & 1 & 2 & 3
    \end{array}
    \quad\quad
    \begin{array}{c|ccccc}
      \times & 0 & 1 & 2 & 3 & 4 \\
      \hline
      0 & 0 & 0 & 0 & 0 & 0 \\
      1 & 0 & 1 & 2 & 3 & 4 \\
      2 & 0 & 2 & 4 & 1 & 3 \\
      3 & 0 & 3 & 1 & 4 & 2 \\
      4 & 0 & 4 & 3 & 2 & 1
    \end{array}
  \]
\end{ejemplo}

\begin{ejemplo}
  He aquí las tablas de adición y multiplicación módulo $6$:
  \[
    \begin{array}{c|cccccc}
      + & 0 & 1 & 2 & 3 & 4 & 5 \\
      \hline
      0 & 0 & 1 & 2 & 3 & 4 & 5 \\
      1 & 1 & 2 & 3 & 4 & 5 & 0 \\
      2 & 2 & 3 & 4 & 5 & 0 & 1 \\
      3 & 3 & 4 & 5 & 0 & 1 & 2 \\
      4 & 4 & 5 & 0 & 1 & 2 & 3 \\
      5 & 5 & 0 & 1 & 2 & 3 & 4
    \end{array}
    \quad\quad
    \begin{array}{c|cccccc}
      \times & 0 & 1 & 2 & 3 & 4 & 5 \\
      \hline
      0 & 0 & 0 & 0 & 0 & 0 & 0 \\
      1 & 0 & 1 & 2 & 3 & 4 & 5 \\
      2 & 0 & 2 & 4 & 0 & 2 & 4 \\
      3 & 0 & 3 & 0 & 3 & 0 & 3 \\
      4 & 0 & 4 & 2 & 0 & 4 & 2 \\
      5 & 0 & 5 & 4 & 3 & 2 & 1
    \end{array}
  \]
\end{ejemplo}

\begin{problema}
  Compile las tablas de adición y multiplicación mód $n = 7, 8$.
\end{problema}

\begin{problema}
  Demuestre que las ecuaciones
  $$3x^2 + 2 = y^2, \quad 7x^3 + 2 = y^3$$
  no tienen soluciones $x,y \in \ZZ$, usando reducción módulo algunos $p$.
\end{problema}

\begin{problema}
  \label{probl:pequeno-Fermat}
  Sea $p$ un número primo.

  \begin{enumerate}
  \item[a)] Demuestre que $p\mid {p\choose k}$ para $k = 1,2,\ldots,p-1$.

  \item[b)] Deduzca de a) que ${p-1\choose k} \equiv (-1)^k \pmod{p}$.

  \item[c)] Deduzca de a) la <<fórmula del binomio mód $p$>>:
    $$(a+b)^p \equiv a^p + b^p \pmod{p}$$
    para cualesquiera $a,b \in \ZZ$.

  \item[d)] Deduzca de c) el \textbf{pequeño teorema de Fermat}:
    $a^p \equiv a \pmod{p}$ para todo $a \in \ZZ$.

    Sugerencia: use inducción con caso base $a = 0$ y el paso inductivo mediante
    $(a+1)^p \equiv a^p + 1 \pmod{p}$.
  \end{enumerate}
\end{problema}

\begin{problema}
  Demuestre las siguientes congruencias mód $p$:
  \begin{align*}
    1 + 2 + 3 + \cdots + (p-1) & \equiv 0 \text{ para }p \ne 2,\\
    1^2 + 2^2 + 3^2 + \cdots + (p-1)^2 & \equiv 0 \text{ para }p \ne 2,3,\\
    1^3 + 2^3 + 3^3 + \cdots + (p-1)^3 & \equiv 0 \text{ para }p \ne 2.
  \end{align*}
\end{problema}

\end{document}
