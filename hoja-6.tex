\documentclass{article}

\usepackage{fullpage}

\usepackage[utf8]{inputenc}
\usepackage[spanish]{babel}

\usepackage[T1]{fontenc}
\usepackage{fourier}
\usepackage{paratype}

\usepackage{framed}
\usepackage{hyperref}
\usepackage{amsmath,amssymb}
\usepackage{tikz}

\newcommand{\ZZ}{\mathbb{Z}}
\DeclareMathOperator{\mcd}{mcd}
\DeclareMathOperator{\mcm}{mcm}
\DeclareMathOperator{\ord}{ord}

\usepackage{amsthm}
\theoremstyle{definition}
\newtheorem{problema}{Problema}[section]
\newtheorem*{comentario}{Comentario}
\newtheorem*{definicion}{Definición}
\newtheorem*{ejemplo}{Ejemplo}

\newenvironment{solucion}{\begin{proof}[Solución]\small}{\end{proof}}

\renewcommand*{\thefootnote}{\fnsymbol{footnote}}


\title{Hoja 6: Raíces primitivas mód $p$}
\author{Alexey Beshenov (cadadr@gmail.com)}
\date{23 de septiembre de 2021}

\begin{document}

\maketitle

\setcounter{section}{6}

Un resultado importante sobre los residuos módulo $p$ es el siguiente.

\begin{framed}
  Para todo primo $p$ existe un elemento $x \in (\ZZ/p\ZZ)^\times$ con
  $\ord (x) = p-1 = \phi(p) = \# (\ZZ/p\ZZ)^\times$. En otras palabras,
  $$(\ZZ/p\ZZ)^\times = \{ 1, x, x^2, \ldots, x^{p-2} \}.$$
  Este $x$ se llama una \textbf{raíz primitiva} mód $p$.
\end{framed}

\begin{ejemplo}
  Para $p = 13$ como una raíz primitiva funciona $x = [2]_{13}$:
  \[
    2^0 \equiv 1, ~
    2^1 \equiv 2, ~
    2^2 \equiv 4, ~
    2^3 \equiv 8, ~
    2^4 \equiv 3, ~
    2^5 \equiv 6, ~
    2^6 \equiv 12, ~
    2^7 \equiv 11, ~
    2^8 \equiv 9, ~
    2^9 \equiv 5, ~
    2^{10} \equiv 10, ~
    2^{11} \equiv 7.
  \]
\end{ejemplo}

\begin{ejemplo}
  Módulo $15$ hay $8$ elementos invertibles, y sus ordenes son los siguientes:
  \begin{center}
    \begin{tabular}{rcccccccc}
      \hline
      $a$ & $1$ & $2$ & $4$ & $7$ & $8$ & $11$ & $13$ & $14$ \\
      \hline
      $\ord [a]_{15}$ & $1$ & $4$ & $2$ & $4$ & $4$ & $2$ & $4$ & $2$ \\
      \hline
    \end{tabular}
  \end{center}
  Entonces, no hay elemento $x$ tal que
  $(\ZZ/15\ZZ)^\times = \{ 1, x, x^2, \ldots, x^7 \}$.
  Esto sucede porque $15$ es compuesto.
\end{ejemplo}

En estas notas \textbf{no} vamos a probar la existencia de raíz primitiva.
No es muy difícil, pero el argumento es un poco técnico y nos llevaría un poco
lejos. Lo que pasa es que la prueba no es constructiva, y en general no existe
una fórmula sencilla que para un primo $p$ dé una raíz primitiva mód $p$.
He aquí una pequeña lista de raíces primitivas módulo los primeros diez primos:

\begin{align*}
  p = 2\colon & 1, \\
  p = 3\colon & 2, \\
  p = 5\colon & 2, 3, \\
  p = 7\colon & 3, 5, \\
  p = 11\colon & 2, 6, 7, 8, \\
  p = 13\colon & 2, 6, 7, 11, \\
  p = 17\colon & 3, 5, 6, 7, 10, 11, 12, 14, \\
  p = 19\colon & 2, 3, 10, 13, 14, 15, \\
  p = 23\colon & 5, 7, 10, 11, 14, 15, 17, 19, 20, 21, \\
  p = 29\colon & 2, 3, 8, 10, 11, 14, 15, 18, 19, 21, 26, 27, \\
              & \cdots
\end{align*}

En el resto de problemas, $p$ es un número primo, y se puede asumir existencia
de una raíz primitiva $x \in (\ZZ/p\ZZ)^\times$.

\begin{problema}
  Demuestre que para cada $d \mid (p-1)$, en $(\ZZ/p\ZZ)^\times$ hay exactamente
  $\phi(d)$ elementos de orden $d$.
\end{problema}

\begin{comentario}
  En particular, el problema anterior nos dice que hay $\phi (p-1)$ diferentes
  raíces primitivas mód $p$. El número $\phi (p-1)$ no es tan pequeño respecto a
  $p-1$, así que en práctica, para encontrar una raíz primitiva mód $p$,
  se puede escoger un número $1 < a < p-1$ al azar, y luego comprobar si
  $\ord [a]_p = p-1$.
\end{comentario}

\begin{ejemplo}
  He aquí los ordenes de los residuos mód $p = 13$:

  \begin{center}
    \begin{tabular}{rccccccccccccc}
      \hline
      $a$: & $1$ & $2$ & $3$ & $4$ & $5$ & $6$ & $7$ & $8$ & $9$ & $10$ & $11$ & $12$ \\
      \hline
      $\ord [a]_{13}$: & $1$ & $12$ & $3$ & $6$ & $4$ & $12$ & $12$ & $4$ & $3$ & $6$ & $12$ & $2$ \\
      \hline
    \end{tabular}
  \end{center}
\end{ejemplo}

\begin{problema}
  ~

  \begin{enumerate}
  \item[a)] Demuestre que si $x$, $x'$ son dos raíces primitivas mód $p$,
    entonces $x x'$ no es una raíz primitiva mód $p$.

  \item[b)] Demuestre que si $x$ es una raíz primitiva mód $p$, entonces
    $x^{-1}$ es también una raíz primitiva mód $p$.
  \end{enumerate}
\end{problema}

\begin{problema}
  Sea $p$ un primo impar. Demuestre que existe $a \in \ZZ$ tal que
  $a^2 \equiv -1 \pmod{p}$ si y solamente si $p \equiv 1 \pmod{4}$.
\end{problema}

\begin{problema}[Euler]
  Sea $p$ un primo impar. Demuestre que para $p \nmid a$ se tiene
  la congruencia mód $p$
  \[ a^{\frac{p-1}{2}} \equiv
    \begin{cases}
      +1, & a \text{ es cuadrado mód }p, \\
      -1, & a \text{ no es cuadrado mód }p.
    \end{cases} \]
\end{problema}

\begin{problema}
  Investigue para cuáles primos $p$ existe $x \in \ZZ/p\ZZ$, tal que $x^3 = 1$ y
  $x \ne 0$.
\end{problema}

\begin{problema}
  Sea $p$ un número primo.

  \begin{enumerate}
  \item[a)] Si $p \equiv 1 \pmod{4}$, demuestre que $a$ es una raíz primitiva
    módulo $p$ si y solamente si $-a$ lo es.

  \item[b)] Si $p \equiv 3 \pmod{4}$, demuestre que $a$ es una raíz primitiva
    módulo $p$ si y solamente si $-a$ tiene orden $\frac{p-1}{2}$.
  \end{enumerate}
\end{problema}

\begin{problema}
  Demuestre que
  $$1^k + 2^k + \cdots + (p-1)^k \equiv 0 \pmod{p}$$
  si $p-1 \nmid k$. Por ejemplo,
  $$1^3 + 2^3 + 3^3 + 4^3 = 100 \equiv 0 \pmod{5}.$$
\end{problema}

\begin{problema}[Gauss]
  Demuestre que si $a_1, \ldots, a_s$ son diferentes raíces primitivas módulo
  $p$, entonces
  $$a_1 \cdots a_s \equiv 1 \pmod{p}.$$
\end{problema}

\begin{problema}[Teorema de Wilson-3]
  Use la existencia de raíz para probar que $(p-1)! \equiv -1 \pmod{p}$.
\end{problema}

\begin{problema}
  Sea $p$ un primo impar.

  \begin{enumerate}
  \item[a)] Demuestre que en $(\ZZ/p\ZZ)^\times$ hay exactamente $\frac{p-1}{2}$
    cuadrados (elementos de la forma $x^2$ para $x \in (\ZZ/p\ZZ)^\times$).

  \item[b)] Demuestre que los conjuntos $X = \{ x^2 \mid x \in \ZZ/p\ZZ \}$ e
    $Y = \{ -1-y^2 \mid y \in \ZZ/p\ZZ \}$ tienen intersección no vacía.

  \item[c)] Deduzca que siempre existen $m,n \in \ZZ$ tales que
    $m^2 + n^2 + 1 \equiv 0 \pmod{p}$.
  \end{enumerate}
\end{problema}

\end{document}
