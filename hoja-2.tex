\documentclass{article}

\usepackage{fullpage}

\usepackage[utf8]{inputenc}
\usepackage[spanish]{babel}

\usepackage[T1]{fontenc}
\usepackage{fourier}
\usepackage{paratype}

\usepackage{framed}
\usepackage{hyperref}
\usepackage{amsmath,amssymb}
\usepackage{tikz}

\newcommand{\ZZ}{\mathbb{Z}}
\DeclareMathOperator{\mcd}{mcd}
\DeclareMathOperator{\mcm}{mcm}
\DeclareMathOperator{\ord}{ord}

\usepackage{amsthm}
\theoremstyle{definition}
\newtheorem{problema}{Problema}[section]
\newtheorem*{comentario}{Comentario}
\newtheorem*{definicion}{Definición}
\newtheorem*{ejemplo}{Ejemplo}

\newenvironment{solucion}{\begin{proof}[Solución]\small}{\end{proof}}

\renewcommand*{\thefootnote}{\fnsymbol{footnote}}


\title{Hoja 2: Teorema chino del residuo}
\author{Alexey Beshenov (cadadr@gmail.com)}
\date{23 de septiembre de 2021}

\begin{document}

\maketitle
\thispagestyle{empty}

\setcounter{section}{2}

\begin{problema}[Teorema chino del residuo]
  \label{probl:CRT}
  Sean $m$ y $n$ dos enteros coprimos ($\mcd (m,n) = 1$).

  \begin{itemize}
  \item[a)] Demuestre que existen enteros $e_1$, $e_2$ tales que
    \begin{gather*}
      e_1 \equiv 1 \pmod{m}, \quad e_1 \equiv 0 \pmod{n}, \\
      e_2 \equiv 0 \pmod{m}, \quad e_2 \equiv 1 \pmod{n}.
    \end{gather*}
    Sugerencia: identidad de Bézout.

  \item[b)] Dados $a, b \in \ZZ$, encuentre $c$ tal que
    $$c \equiv a \pmod{m}, \quad c \equiv b \pmod{n}.$$
  \end{itemize}
\end{problema}

Podemos reformular el resultado de arriba de la siguiente manera.

\begin{framed}
  Para $\mcd (m,n) = 1$ la aplicación
  \begin{align*}
    \Phi\colon \ZZ/mn\ZZ & \to \ZZ/m\ZZ \times \ZZ/n\ZZ,\\
    [c]_{mn} & \mapsto ([c]_m, \, [c]_n)
  \end{align*}
  es biyectiva.
\end{framed}

\begin{proof}
  Acabamos de probar que para todo par de residuos $([a]_m, [b]_n)$ habrá
  $[c]_{mn}$ tal que $\Phi ([c]_{mm}) = ([a]_m,[b]_n)$.  Además, usando que
  $$|\ZZ/mn\ZZ| = |\ZZ/m\ZZ \times \ZZ/n\ZZ| = |\ZZ/m\ZZ| \times |\ZZ/n\ZZ|,$$
  podemos concluir que $\Phi$ es también inyectiva.
\end{proof}

\begin{ejemplo}
  He aquí la aplicación $\Phi$ para $(m,n) = (2,3)$:
  \begin{gather*}
    \Phi\colon \ZZ/6\ZZ \to \ZZ/2\ZZ \times \ZZ/3\ZZ,\\
    0 \mapsto (0,0), ~
    1 \mapsto (1,1), ~
    2 \mapsto (0,2), ~
    3 \mapsto (1,0), ~
    4 \mapsto (0,1), ~
    5 \mapsto (1,2).
  \end{gather*}
\end{ejemplo}

\begin{problema}
  Para $(m,n) = (3,5)$ reduzca los residuos mód $15$ módulo $3$ y $5$:
  \begin{align*}
    \ZZ/15\ZZ & \to \ZZ/3\ZZ \times \ZZ/5\ZZ, \\
    0 & \mapsto (0,0), \\
    1 & \mapsto (1,1), \\
              & \ldots \\
    10 & \mapsto (1,0), \\
      & \ldots
  \end{align*}
  Verifique que en la parte derecha aparecen todos los pares de residuos
  $([a]_3, [b]_5)$ para $a = 0,1,2$ y $b = 0,1,2,3,4$.
\end{problema}

\begin{ejemplo}
  Vamos a resolver la congruencia $x^2 \equiv 1 \pmod{40}$.
  Hay cuatro (!)  soluciones mód $8$:
  $$1^2 \equiv 3^2 \equiv 5^2 \equiv 7^2 \equiv 1 \pmod{8}.$$
  Por otra parte, mód $5$ hay solo dos soluciones esperadas $x \equiv \pm 1$.
  El teorema chino del resto nos permite concluir que mód $40$ hay ocho
  soluciones:
    \begin{align*}
      \ZZ/40\ZZ & \to \ZZ/8\ZZ \times \ZZ/5\ZZ, \\
      1 & \mapsto (1, 1), \\
      9 & \mapsto (1, 4), \\
      11 & \mapsto (3, 1), \\
      19 & \mapsto (3, 4), \\
      21 & \mapsto (5, 1), \\
      29 & \mapsto (5, 4), \\
      31 & \mapsto (7, 1), \\
      39 & \mapsto (7, 4).
    \end{align*}

  ¿Cómo reconstruir las soluciones mód $40$?
  ¡Esto nos diría la solución del problema \ref{probl:CRT}!

  Usando $\mcd(8,5) = 1$, escribamos la identidad de Bézout:
  $$2\cdot 8 + (-3)\cdot 5 = 1.$$
  De aquí se ve que
  \begin{align*}
    16 & \equiv 0 \pmod{8}, & \quad 16 & \equiv 1 \pmod{5}, \\
    -15 & \equiv 1 \pmod{8}, & \quad -15 & \equiv 0 \pmod{5}.
  \end{align*}

  Por ejemplo, buscamos $x$ tal que
  $$x \equiv 5 \pmod{8}, \quad x \equiv 4 \pmod{5}.$$
  Entonces,
  $$x = 5\cdot (-15) + 4\cdot 16 = -11 \equiv 29 \pmod{40}.$$
\end{ejemplo}

\begin{problema}
  \label{probl:crt-invertibles}
  Demuestre que la biyección
  $$\Phi\colon \ZZ/mn\ZZ \to \ZZ/m\ZZ \times \ZZ/n\ZZ$$
  se restringe a una biyección entre los elementos invertibles
  \begin{align*}
    (\ZZ/mn\ZZ)^\times & \to (\ZZ/m\ZZ)^\times \times (\ZZ/n\ZZ)^\times, \\
    [a]_{mn} & \mapsto ([a]_m, [a]_n).
  \end{align*}
\end{problema}

\begin{problema}
  Formule y demuestre una versión del teorema chino del residuo para módulos
  $n_1, \ldots, n_s$ tales que $\mcd (n_i,n_j) = 1$ para $i \ne j$.
\end{problema}

\begin{problema}
  Sea $f (x) = a_m\,x^m + \cdots + a_1\,x + a_0$ un polinomio con coeficientes
  enteros.

  \begin{enumerate}
  \item[a)] Demuestre que la congruencia $f (x) \equiv 0 \pmod{n}$ tiene
    solución para $n = p_1^{e_1} \cdots p_s^{e_s}$ si y solamente si
    $f (x) \equiv 0 \pmod{p_i^{e_i}}$ tiene solución para cada $i = 1,\ldots,s$.

  \item[b)] Sea $N$ el número de soluciones de $f (x) \equiv 0 \pmod{n}$ y $N_i$
    el número de soluciones de $f (x) \equiv 0 \pmod{p_i^{e_i}}$. Demuestre que
    $N = N_1 \cdots N_s$.
  \end{enumerate}
\end{problema}

\begin{problema}
  ~

  \begin{enumerate}
  \item[a)] Demuestre que $x^2 \equiv x \pmod{p^e}$ tiene únicas soluciones
    $x = 0$ y $1$ para todo primo $p$ y $e = 1,2,3,\ldots$

  \item[b)] En general, ¿cuántas soluciones tiene $x^2 \equiv x \pmod{n}$?
  \end{enumerate}
\end{problema}

\begin{problema}
  ~

  \begin{enumerate}
  \item[a)] Demuestre que $x^2 \equiv 1 \pmod{p^e}$ tiene únicas soluciones
    $x = \pm 1$ para $p$ impar y $e = 1,2,3,\ldots$

  \item[b)] Demuestre que $x^2 \equiv 1 \pmod{2^e}$ tiene $4$ soluciones para
    $e \ge 3$. ¿Cuáles son?

  \item[c)] En general, ¿cuántas soluciones tiene $x^2 \equiv 1 \pmod{n}$?
  \end{enumerate}
\end{problema}

\begin{problema}
  Use el teorema chino del residuo para encontrar las soluciones de
  $x^2 \equiv x$ y $x^2 \equiv 1$ mód $n = 221$.
\end{problema}

\end{document}
