\documentclass{article}

\usepackage{fullpage}

\usepackage[utf8]{inputenc}
\usepackage[spanish]{babel}

\usepackage[T1]{fontenc}
\usepackage{fourier}
\usepackage{paratype}

\usepackage{framed}
\usepackage{hyperref}
\usepackage{amsmath,amssymb}
\usepackage{tikz}

\newcommand{\ZZ}{\mathbb{Z}}
\DeclareMathOperator{\mcd}{mcd}
\DeclareMathOperator{\mcm}{mcm}
\DeclareMathOperator{\ord}{ord}

\usepackage{amsthm}
\theoremstyle{definition}
\newtheorem{problema}{Problema}[section]
\newtheorem*{comentario}{Comentario}
\newtheorem*{definicion}{Definición}
\newtheorem*{ejemplo}{Ejemplo}

\newenvironment{solucion}{\begin{proof}[Solución]\small}{\end{proof}}

\renewcommand*{\thefootnote}{\fnsymbol{footnote}}


\title{Hoja 3: Función $\phi$ de Euler}
\author{Alexey Beshenov (cadadr@gmail.com)}
\date{23 de septiembre de 2021}

\begin{document}

\maketitle

\setcounter{section}{3}

\begin{definicion}
  Para un número natural $n$, la \textbf{función $\phi$ de Euler} es el número
  de los residuos invertibles mód $n$:
  $$\phi (n) = \# (\ZZ/n\ZZ)^\times.$$
  O de manera equivalente (véase el problema~0.3),
  es el número de $1 \le a < n$ coprimos con $n$:
  $$\phi (n) = \# \{ 1 \le a < n \mid \mcd (a,n) = 1 \}.$$
\end{definicion}

\begin{ejemplo}
  He aquí algunos valores de $\phi (n)$:

  \begin{center}
    \begin{tabular}{rcccccccccccccccccccc}
      \hline
      $n\colon$ & $1$& $2$& $3$& $4$& $5$& $6$& $7$& $8$& $9$& $10$& $11$& $12$& $13$& $14$& $15$& $16$& $17$& $18$& $19$& $20$ \\
      $\phi(n)\colon$ & $1$& $1$& $2$& $2$& $4$& $2$& $6$& $4$& $6$& $4$& $10$& $4$& $12$& $6$& $8$& $8$& $16$& $6$& $18$& $8$ \\
      \hline
      $n\colon$ & $21$& $22$& $23$& $24$& $25$& $26$& $27$& $28$& $29$& $30$& $31$& $32$& $33$& $34$& $35$& $36$& $37$& $38$& $39$& $40$ \\
      $\phi(n)\colon$ & $12$& $10$& $22$& $8$& $20$& $12$& $18$& $12$& $28$& $8$& $30$& $16$& $20$& $16$& $24$& $12$& $36$& $18$& $24$& $16$ \\
      \hline
      $n\colon$ & $41$& $42$& $43$& $44$& $45$& $46$& $47$& $48$& $49$& $50$& $51$& $52$& $53$& $54$& $55$& $56$& $57$& $58$& $59$& $60$ \\
      $\phi(n)\colon$ & $40$& $12$& $42$& $20$& $24$& $22$& $46$& $16$& $42$& $20$& $32$& $24$& $52$& $18$& $40$& $24$& $36$& $28$& $58$& $16$ \\
      \hline
      $n\colon$ & $61$& $62$& $63$& $64$& $65$& $66$& $67$& $68$& $69$& $70$& $71$& $72$& $73$& $74$& $75$& $76$& $77$& $78$& $79$& $80$ \\
      $\phi(n)\colon$ & $60$& $30$& $36$& $32$& $48$& $20$& $66$& $32$& $44$& $24$& $70$& $24$& $72$& $36$& $40$& $36$& $60$& $24$& $78$& $32$ \\
      \hline
      $n\colon$ & $81$& $82$& $83$& $84$& $85$& $86$& $87$& $88$& $89$& $90$& $91$& $92$& $93$& $94$& $95$& $96$& $97$& $98$& $99$& $100$ \\
      $\phi(n)\colon$ & $54$& $40$& $82$& $24$& $64$& $42$& $56$& $40$& $88$& $24$& $72$& $44$& $60$& $46$& $72$& $32$& $96$& $42$& $60$& $40$ \\
      \hline
    \end{tabular}
  \end{center}
\end{ejemplo}

\begin{problema}
  \label{probl:phi-para-primario}
  Calcule que para un primo $p$ y $e = 1,2,3,\ldots$ se tiene
  $$\phi (p^e) = p^e - p^{e-1} = p^e\,\left(1 - \frac{1}{p}\right).$$
\end{problema}

\begin{problema}[Multiplicatividad]
  \label{probl:phi-multiplicativo}
  Demuestre que si $\mcd (m,n) = 1$, entonces
  $$\phi (mn) = \phi(m)\,\phi(n).$$

  Sugerencia: véase el problema~2.2.
\end{problema}

\begin{problema}
  Deduzca de \ref{probl:phi-para-primario} y \ref{probl:phi-multiplicativo} la
  fórmula
  $$\phi (n) = n\,\prod_{p \mid n} \left(1 - \frac{1}{p}\right),$$
  donde el producto es sobre todos los divisores primos de $n$.
\end{problema}

\begin{problema}
  ~

  \begin{itemize}
  \item[a)] Demuestre que $\phi(n)$ es par para $n \ge 3$.

  \item[b)] ¿Para cuáles $n$ se tiene $\phi (n) \le 10$?

  \item[c)] ¿Para cuáles $n$ se tiene $\phi (n) = 100$?
  \end{itemize}
\end{problema}

\begin{problema}
  Demuestre que $\phi (n) \le n-1$ para $n \ge 2$, y la igualdad se cumple si y
  solo si $n = p$ es primo.
\end{problema}

\begin{problema}[Congruencia de Euler]
  \label{probl:congruencia-de-euler}
  Sean
  \[ (\ZZ/n\ZZ)^\times = \{ x_1, x_2, \ldots, x_{\phi (n)} \} \]
  los residuos invertibles mód $n$.

  \begin{enumerate}
  \item[a)] Demuestre que si $x$ es también invertible, entonces
    $$\{ x a_1, x x_2, \ldots, x x_{\phi (n)} \} = (\ZZ/n\ZZ)^\times.$$

  \item[b)] Use el punto anterior para probar la congruencia de Euler:
    $x^{\phi(n)} = 1$ para $x \in (\ZZ/n\ZZ)^\times$,
    o de manera equivalente:
    $$a^{\phi (n)} \equiv 1 \pmod{n} \quad \text{para }\mcd (a,n) = 1.$$
  \end{enumerate}
\end{problema}

Note que la congruencia de Euler generaliza el pequeño teorema de Fermat.

\begin{problema}
  Demuestre las siguientes identidades para cualesquiera $m,n$:

  \begin{enumerate}
  \item[a)] $\phi (mn) = \phi (m)\,\phi (n) \, \frac{d}{\phi (d)}$,
    donde $d = \mcd (m,n)$.

  \item[b)] $\phi(\mcd (m,n)) \, \phi(\mcm (m,n)) = \phi(m)\,\phi(n)$.
  \end{enumerate}
\end{problema}

\begin{problema}[Gauss]
  Para $n \ge 1$ consideremos las fracciones
  \[
    \frac{1}{n}, ~
    \frac{2}{n}, ~
    \frac{3}{n}, ~
    \ldots, ~
    \frac{n-1}{n}, ~
    \frac{n}{n}.
  \]
  Luego escribamos cada una de la forma $\frac{a}{b}$ con $\mcd(a,b) = 1$.

  \begin{enumerate}
  \item[a)] Demuestre que para cada $d \mid n$, el número de fracciones en la
    lista con $d$ en el denominador es precisamente $\phi (d)$.

  \item[b)] Deduzca la identidad $\sum_{d\mid n} \phi (d) = n$, donde la suma es
    sobre todos los divisores de $n$.
  \end{enumerate}
\end{problema}

\begin{ejemplo}
  Para $n = 12$ tenemos
  \[ \phi (1) + \phi (2) + \phi (3) + \phi (4) + \phi (6) + \phi (12) =
    1 + 1 + 2 + 2 + 2 + 4 = 12. \]
\end{ejemplo}

\begin{problema}
  Demuestre que un primo $p$ satisface $\phi (p) = 2\,\phi (p-1)$ si y solamente
  si $p = 2^{2^k} + 1$ para algún $k$ (es decir, es un primo de Fermat).
\end{problema}

\begin{problema}
  Calcule la suma de los números coprimos con $n$:
  $$\sum_{\substack{1 \le t \le n \\ \mcd (t,n) = 1}} t = \frac{\phi(n)}{2}\,n.$$

  Sugerencia: escriba la suma de dos maneras:
  \[
    \sum_{\substack{1 \le t \le n \\ \gcd (t,n) = 1}} t =
    \sum_{\substack{1 \le n-t \le n \\ \gcd (n-t,n) = 1}} (n-t) =
    \sum_{\substack{1 \le t \le n \\ \gcd (t,n) = 1}} (n-t).
  \]
\end{problema}

\end{document}
